%!TeX root=../tese.tex
%("dica" para o editor de texto: este arquivo é parte de um documento maior)
% para saber mais: https://tex.stackexchange.com/q/78101/183146

% Apague as duas linhas abaixo (elas servem apenas para gerar um
% aviso no arquivo PDF quando não há nenhum dado a imprimir) e
% copie, com as alterações necessárias, o conteúdo do arquivo
% conteudo-exemplo/paginas-preliminares.tex

%%%%%%%%%%%%%%%%%%%% DEDICATÓRIA, RESUMO, AGRADECIMENTOS %%%%%%%%%%%%%%%%%%%%%%%

% \begin{dedicatoria}
% Esta seção é opcional e fica numa página separada; ela pode ser usada para
% uma dedicatória ou epígrafe.
% \end{dedicatoria}

% Reinicia o contador de páginas (a próxima página recebe o número "i") para
% que a página da dedicatória não seja contada.
% \pagenumbering{roman}

% Agradecimentos:
% Se o candidato não quer fazer agradecimentos, deve simplesmente eliminar
% esta página. A epígrafe, obviamente, é opcional; é possível colocar
% epígrafes em todos os capítulos. O comando "\chapter*" faz esta seção
% não ser incluída no sumário.
\chapter*{Agradecimentos}
Primeiramente, agradeço ao João Eduardo Ferreira pela oportunidade de participar do projeto Hackers do Bem e por sua orientação durante o desenvolvimento deste trabalho. Aos integrantes do IMEsec pelo conhecimento e experiências adquiridas durante minha participação. Aos integrantes do Hackers do Bem pela ajuda e apoio no desenvolvimento deste trabalho. E também a minha família por todo apoio que me dão.
% \epigrafe{Do. Or do not. There is no try.}{Mestre Yoda}

% Resumo e abstract são definidos no arquivo "metadados.tex". Este
% comando também gera automaticamente a referência para o próprio
% documento, conforme as normas sugeridas da USP.
\printResumoAbstract


%%%%%%%%%%%%%%%%%%%%%%%%%%% LISTAS DE FIGURAS ETC. %%%%%%%%%%%%%%%%%%%%%%%%%%%%%

% Como as listas que se seguem podem não incluir uma quebra de página
% obrigatória, inserimos uma quebra manualmente aqui.
\makeatletter
\if@openright\cleardoublepage\else\clearpage\fi
\makeatother

% Todas as listas são opcionais; Usando "\chapter*" elas não são incluídas
% no sumário. As listas geradas automaticamente também não são incluídas por
% conta das opções "notlot" e "notlof" que usamos para a package tocbibind.

% Normalmente, "\chapter*" faz o novo capítulo iniciar em uma nova página, e as
% listas geradas automaticamente também por padrão ficam em páginas separadas.
% Como cada uma destas listas é muito curta, não faz muito sentido fazer isso
% aqui, então usamos este comando para desabilitar essas quebras de página.
% Se você preferir, comente as linhas com esse comando e des-comente as linhas
% sem ele para criar as listas em páginas separadas. Observe que você também
% pode inserir quebras de página manualmente (com \clearpage, veja o exemplo
% mais abaixo).
\newcommand\disablenewpage[1]{{\let\clearpage\par\let\cleardoublepage\par #1}}

% Nestas listas, é melhor usar "raggedbottom" (veja basics.tex). Colocamos
% a opção correspondente e as listas dentro de um grupo para ativar
% raggedbottom apenas temporariamente.
\bgroup
\raggedbottom

%%%%% Listas criadas manualmente

%\chapter*{Lista de Abreviaturas}
\disablenewpage{\chapter*{Lista de Abreviaturas}}

\begin{tabular}{rl}
   SQL & Linguagem de Consulta Estruturada (\emph{Structured Query Language})\\
  HTTP & Protocolo de Transferência de Hipertexto (\emph{Hypertext Transfer Protocol})\\
   HDB & Potencial de interação elétron-íon (\emph{Electron-Ion Interaction Potentials})\\
 VUMOS & Sistema de Monitoramento de vulnerabilidades (\emph{Vulnerability Monitoring System})\\
   URL & Localizador Uniforme de Recursos (\emph{Uniform Resource Locator})\\
   TLS & Segurança de Camada de Transporte (\emph{Transport Layer Security})\\
  CIRD & Roteamento entre domínios sem classes (\emph{Classless Inter-Domain Routing})\\
   RCE & Execução de código remoto (\emph{Remote Code Execution})\\
   IME & Instituto de Matemática e Estatística\\
   USP & Universidade de São Paulo
\end{tabular}

%\chapter*{Lista de Símbolos}

% \disablenewpage{\chapter*{Lista de Símbolos}}

% \begin{tabular}{rl}
%   $\omega$ & Frequência angular\\
%     $\psi$ & Função de análise \emph{wavelet}\\
%     $\Psi$ & Transformada de Fourier de $\psi$\\
% \end{tabular}

% Quebra de página manual
\clearpage

%%%%% Listas criadas automaticamente

% Você pode escolher se quer ou não permitir a quebra de página
%\listoffigures
\disablenewpage{\listoffigures}

% Você pode escolher se quer ou não permitir a quebra de página
%\listoftables
\disablenewpage{\listoftables}

% Esta lista é criada "automaticamente" pela package float quando
% definimos o novo tipo de float "program" (em utils.tex)
% Você pode escolher se quer ou não permitir a quebra de página
%\listof{program}{\programlistname}
\disablenewpage{\listof{program}{\programlistname}}

% Sumário (obrigatório)
\tableofcontents

\egroup % Final de "raggedbottom"

% Referências indiretas ("x", veja "y") para o índice remissivo (opcionais,
% pois o índice é opcional). É comum colocar esses itens no final do documento,
% junto com o comando \printindex, mas em alguns casos isso torna necessário
% executar texindy (ou makeindex) mais de uma vez, então colocar aqui é melhor.
\index{Inglês|see{Língua estrangeira}}
\index{Figuras|see{Floats}}
\index{Tabelas|see{Floats}}
\index{Código-fonte|see{Floats}}
\index{Subcaptions|see{Subfiguras}}
\index{Sublegendas|see{Subfiguras}}
\index{Equações|see{Modo Matemático}}
\index{Fórmulas|see{Modo Matemático}}
\index{Rodapé, notas|see{Notas de rodapé}}
\index{Captions|see{Legendas}}
\index{Versão original|see{Tese/Dissertação, versões}}
\index{Versão corrigida|see{Tese/Dissertação, versões}}
\index{Palavras estrangeiras|see{Língua estrangeira}}
\index{Floats!Algoritmo|see{Floats, Ordem}}
