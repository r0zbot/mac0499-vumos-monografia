%!TeX root=../tese.tex
%("dica" para o editor de texto: este arquivo é parte de um documento maior)
% para saber mais: https://tex.stackexchange.com/q/78101/183146

% Insira aqui os metadados do seu trabalho. Para isso, copie,
% com as alterações necessárias, o conteúdo do arquivo
% conteudo-exemplo/metadados.tex

%!TeX root=../tese.tex
%("dica" para o editor de texto: este arquivo é parte de um documento maior)
% para saber mais: https://tex.stackexchange.com/q/78101/183146

%%%%%%%%%%%%%%%%%%%%%%%%%%%%%%%%%%%%%%%%%%%%%%%%%%%%%%%%%%%%%%%%%%%%%%%%%%%%%%%%
%%%%%%%%%%%%%%%%%%%%%%%%%%%%% METADADOS DA TESE %%%%%%%%%%%%%%%%%%%%%%%%%%%%%%%%
%%%%%%%%%%%%%%%%%%%%%%%%%%%%%%%%%%%%%%%%%%%%%%%%%%%%%%%%%%%%%%%%%%%%%%%%%%%%%%%%

% Estes comandos definem o título e autoria do trabalho e devem sempre ser
% definidos, pois além de serem utilizados para criar a capa, também são
% armazenados nos metadados do PDF.
\title{
    % Obrigatório nas duas línguas
    titlept={VuMoS},
    titleen={VuMoS},
    % Opcional, mas se houver deve existir nas duas línguas
    subtitlept={Um sistema de monitoramento de vulnerabilidades de segurança para os sistemas independentes da USP},
    subtitleen={Security vulnerability monitoring system for the independent systems at USP},
    % Opcional, para o cabeçalho das páginas
    shorttitle={VuMoS: Monitoramento de vulnerabilidades na USP},
}

\author{Daniel de Sousa Martinez}

% Para TCCs, este comando define o supervisor
\orientador{Prof. João Eduardo Ferreira}

% Se não houver, remova; se houver mais de um, basta
% repetir o comando quantas vezes forem necessárias

% A página de rosto da versão para depósito (ou seja, a versão final
% antes da defesa) deve ser diferente da página de rosto da versão
% definitiva (ou seja, a versão final após a incorporação das sugestões
% da banca).
\defesa{
  nivel=tcc, % mestrado, doutorado ou tcc
  % É a versão para defesa ou a versão definitiva?
  %definitiva,
  % É qualificação?
  quali,
  programa={Ciência da Computação},
  membrobanca={Profª. Drª. Fulana de Tal (orientadora) -- IME-USP [sem ponto final]},
  % Em inglês, não há o "ª"
  %membrobanca{Prof. Dr. Fulana de Tal (advisor) -- IME-USP [sem ponto final]},
  membrobanca={Prof. Dr. Ciclano de Tal -- IME-USP [sem ponto final]},
  membrobanca={Profª. Drª. Convidada de Tal -- IMPA [sem ponto final]},
  % Se não houver, remova
  apoio={Durante o desenvolvimento deste trabalho o autor
         recebeu auxílio financeiro da Superintendência de Tecnologia da Informação da USP},
  local={São Paulo},
  data=2021-08-10, % YYYY-MM-DD
  % A licença do seu trabalho. Use CC-BY, CC-BY-NC, CC-BY-ND, CC-BY-SA,
  % CC-BY-NC-SA ou CC-BY-NC-ND para escolher a licença Creative Commons
  % correspondente (o sistema insere automaticamente o texto da licença).
  % Se quiser estabelecer regras diferentes para o uso de seu trabalho,
  % converse com seu orientador e coloque o texto da licença aqui, mas
  % observe que apenas TCCs sob alguma licença Creative Commons serão
  % acrescentados ao BDTA.
  direitos={CC-BY}, % Creative Commons Attribution 4.0 International License
  %direitos={Autorizo a reprodução e divulgação total ou parcial
  %          deste trabalho, por qualquer meio convencional ou
  %          eletrônico, para fins de estudo e pesquisa, desde que
  %          citada a fonte.},
  % Isto deve ser preparado em conjunto com o bibliotecário
  %fichacatalografica={nome do autor, título, etc.},
}

% As palavras-chave são obrigatórias, em português e
% em inglês. Acrescente quantas forem necessárias.
\palavrachave{segurança}
\palavrachave{microsserviços}
\palavrachave{monitoramento}
\palavrachave{vulnerabilidades}

\keyword{security}
\keyword{microservices}
\keyword{monitoring}
\keyword{vulnerabilities}

% TODO: escrever resumo e abstract
% O resumo é obrigatório, em português e inglês.
\resumo{
    Vulnerabilidades de segurança nos sistemas hospedados pela USP podem ocasionar grandes problemas, pois podem expor dados pessoais dos alunos e funcionários, permitir acesso não autorizado a sistemas internos, prejudicar a imagem da universidade, dentre outros. E pela natureza descentralizada do funcionamento da rede da USP, torna-se difícil escanear por vulnerabilidades de forma organizada. Por conta disso, o objetivo deste trabalho foi coletar de maneira automatizada e concentrar informações correspondentes a falhas de segurança em todos os sistemas publicamente expostos da USP. Isso foi feito através do desenvolvimento de um sistema modular, implementado utilizando uma arquitetura de microsserviços, onde cada módulo pode utilizar outras ferramentas existentes para realizar sua função.
}
% TODO: escrever resumo e abstract
\abstract{
    Security vulnerabilities on systems hosted by USP may cause big problems, as they can expose student and staff personal data, allow unauthorized external access to internal systems, or even harm the university's public image. Due to the decentralized nature of the USP's network, it's difficult to scan for vulnerabilities in a coherent way. As such, the objective of this work is to collect in an automated way and concentrate relevant information about security issues in all publicly available USP systems. This was done through the development of a modular system, implemented using a micro services architecture, where each module may use other tools to do their corresponding job.
}
