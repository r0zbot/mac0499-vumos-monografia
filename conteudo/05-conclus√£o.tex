\chapter{Conclusão}
\label{cap:conclusao}

\section{Contribuição para a segurança da universidade}

Pode ser dize que o objetivo inicial do projeto foi atingido, pois agora existe uma plataforma capaz de unificar todos os dados de vulnerabilidades encontradas na universidade.

Além disso, os módulos em execução já encontraram vulnerabilidades que precisam ser corrigidas, e portanto, já beneficiaram a comunidade como um todo.

% 1: Contribuição da ferramenta para a seguranca da universidade

\section{Contribuição para a formação acadêmica}
% 2: Contribuição para a formação Ajudou na minha especialização na área de segurança

Durante o desenvolvimento deste projeto, aprendi muito sobre a arquitetura de microsserviços e como ela pode tanto facilitar quanto dificultar o desenvolvimento de um projeto. 
Também pude me especializar na área de segurança da informação, aprendendo a fundo detalhes sobre diversos sistemas, quais podem ser suas possíveis vulnerabilidades, e também como explorá-las. Isso me traz um conhecimento de como desenvolver aplicações mais seguras, evitando erros que são comumente cometidos.

% \subsection{Contribuição do Hackers do Bem}



% 3: Contribuição da HDB

\section{Próximos passos}

Para os próximos passos do projeto, poderão ser implementados novos módulos, como:
\begin{itemize}
    \item \emph{Cameradar}: Um módulo que escaneia por câmeras com pouca ou nenhuma autenticação.
    \item \emph{Metasploit}: Diversos módulos do metasploit podem ser implementados também, para buscar vulnerabilidades em outros tipos de sistema, como RDP, OpenSSL ou SSH, por exemplo.
    \item \emph{Nmap NSE}: Diversos módulos que utilizam \textit{scripts} do \textit{Nmap Scripting Engine} também podem ser implementados, para que os serviços encontrados sejam também associados a uma CVE, por exemplo.
    \item \emph{Shodan.io}: Um módulo que extrai as informações existentes em \url{shodan.io} também pode ser facilmente implementado, e disponibilizaria uma grande quantidade de informações pré escaneadas. Porém, ele necessita de uma licença para fazer uso completo de sua API.
\end{itemize}

Além disso, podem ser feitas melhorias na interface do usuário, permitindo que \textit{Services}, \textit{Targets} e \textit{Issues} sejam adicionados manualmente para incluir vulnerabilidas encontradas fora da plataforma, por exemplo. 

Isso pode ser feito por novas gerações de integrantes do projeto \textit{Hackers do Bem}, pois grande parte do trabalho que antes era feito manualmente pelo grupo pode ser automatizado de forma relativamente simples usando o VuMoS.