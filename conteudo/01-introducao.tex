%!TeX root=../tese.tex
%("dica" para o editor de texto: este arquivo é parte de um documento maior)
% para saber mais: https://tex.stackexchange.com/q/78101/183146

%% ------------------------------------------------------------------------- %%
\chapter{Introdução}
\label{cap:introducao}

\section{Motivação}
% TODO: FIX CITATION

\subsection{Hackers do Bem}
Para melhor entender a motivação por trás desse projeto, é interessante conhecer o grupo responsável pelo seu desenvolvimento. O grupo \textit{Hackers do Bem} foi formado em abril de 2020, motivado em parte pelo vazamento de dados da Faculdade de Medicina da Universidade de São Paulo. Composto por alunos do Bacharelado em Ciência da Computação do Instituto de Matemática e Estatística da USP, e supervisionado pelo Superintendente de Tecnologia da Informação, ele conta atualmente com 4 integrantes: Daniel de Sousa Martinez, o autor deste trabalho; Cainã Setti Galante; Victor Aristóteles Rocha Campos e Victor Seiji Hariki, contando também com a supervisão de João Eduardo Ferreira.

A principal atividade do grupo consiste em acessar diversos sistemas hospedados pela universidade e buscar em cada um deles problemas que podem causar brechas de segurança. Mas além disso, também é importante verificar se os sistemas deveriam estar sendo hospedados publicamente acessíveis em primeiro lugar. Logo, outra atividade do grupo é vasculhar as subredes
% TODO: dicionario de palavras?
dos institutos e descobrir que tipos de serviços estão sendo expostos indevidamente, como impressoras ou servidores de arquivos que permitem acesso sem autenticação. 

% TODO: colocar uma justificativa 

% 
\subsection{O desafio}

Por conta do tamanho da rede da USP \footnote{Mais detalhes em \url{https://ipinfo.io/AS28571}}, 
% TODO: AS28571, , quote? 
que conta atualmente com mais de 180 mil possíveis endereços de IP,  e por ela ser composta por diversas subredes independentes (para cada instituto, por exemplo), sem um controle central, uma abordagem manual para vasculhá-la é inviável. Portanto, inicialmente, o grupo operava por meio de amostragem dos endereços de IP encontrados por ferramentas como NMAP. 
% TODO: NMAP?  


Embora existam ferramentas automatizadas para busca de vulnerabilidades, nenhuma delas é compreensiva e escalável o suficiente para uma rede desse tamanho. O serviço que mais se aproxima é o oferecido pelo 
% TODO: link?
shodan.io,
porém ele não traz informações detalhadas o suficiente sobre alguns tipos de vulnerabilidade mais graves, como 
% TODO: ??
SQL Injection e Cross-Site Scripting. 

Por conta disso, em Janeiro de 2021 iniciou-se o desenvolvimento do VuMoS, cujo nome é um acrônimo para \textit{Vulnerability Monitoring System}, ou em português: Sistema de monitoramento de vulnerabilidades.
% TODO: ??



\section{Objetivo}

O objetivo desse projeto, portanto, é concentrar em uma única plataforma as informações obtidas através de inúmeras ferramentas e serviços, e através disso construir um inventário dos sistemas rodando na USP e suas possíveis vulnerabilidades. Adicionalmente, essa plataforma deve ser escalável para centenas de milhares de entradas, possuir uma arquitetura modular e possibilitar a adição e substituição de módulos sem interromper o funcionamento dos demais, pois é possível que um escaneamento total demore diversas semanas por conta do tamanho da rede. 